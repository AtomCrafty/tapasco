%%%%%%%%%%%%%%%%%%%%%%%%%%%%%%%%%%%%%%%%%%%%%%%%%%%%%%%%%%%%%%%%%%%%%%%%%%%%%%%%
\section{Basic Usage}\label{sec:basic-usage}%
This section aims to give a quick tutorial on the basic usage of \tpc{}.
\tpc{} can be run in \tpchome{} via \gloss{sbt} and provides two general commands:

\begin{itemize}
  \item \code{\bfseries hls} --- Performs high-level synthesis for given architectures and kernels.
  \item \code{\bfseries compose} --- Composes a threadpool for given architectures, platforms and bitstream descriptions.
\end{itemize}

Both commands can be started by \gloss{sbt} and given a list of parameters, for example:
%
\begin{lstlisting}[language=bash]
~$ sbt "hls architecture baseline kernel add"
~$ sbt "compose architecture baseline bitstreamDir mybitstreams"
\end{lstlisting}
%
The parameters are usually pairs of case-insensitive key and value, separated by spaces, can be given in any order and are described in more detail in \tblref{tbl:run-description}.
By default, \tpc{} executes in \emph{batch mode}, i.e., it scans the currently set base directories for platform, architecture, kernel and bitstream descriptions, which must be named \code{platform|\allowbreak architecture|\allowbreak kernel|\allowbreak  bitstream.description}, but may reside in arbitrarily nested subdirectories of their corresponding base directory (cf. \figref{fig:dirs}).
%
\begin{figure}
  \centering%
  \includegraphics[]{tikz/directory-structure}
  \caption{Sample directory structure.}
  \label{fig:dirs}
\end{figure}
%
Each base directory can be changed using a command line argument (see also \tblref{tbl:run-description}):
%
\begin{lstlisting}[language=bash]
~$ sbt "compose archDir otherArchs kernelDir otherKernels/myKernels"
\end{lstlisting}
%
Directories are interpreted as paths relative to \tpchome{}, unless an absolute path is given.
In \emph{batch mode}, \tpc{} will build every combination of architecture-platform-bitstream, if possible in parallel.

\medskip
Since this can easily explode combinatorially, the user can provide filters for each component to select specific architectures, platforms, kernels or bitstreams. E.g., to build the kernel \code{add} for the architecture \code{baseline}:
%
\begin{lstlisting}[language=bash]
~$ sbt "hls architecture baseline kernel add"
\end{lstlisting}
%
\tpc{} will scan the default directory \code{kernel/} for a \code{kernel.description} defining \code{add} and will report an error if it could not be found, or multiple conflicting definitions were found.
See \secref{sec:directory-structure} for more information about the default directory structure.
In this case, the user can also supply the filename of a specific \gloss{Kernel Description}:
%
\begin{lstlisting}[language=bash]
~$ sbt "hls architecture baseline kernel mykernels/mydesc.kd"
\end{lstlisting}
%
If the kernel was found, \tpc{} will continue in the same manner to look for an \gloss{Architecture Description} called \code{baseline}.
The same mechanism also applies to \gloss{Platform Descriptions} and \gloss{Bitstream Descriptions}, example:
To compose a threadpool for the \code{baseline} architecture using the \code{zynq} platform and a composition defined in \code{example.bd} use
%
\begin{lstlisting}[language=bash]
~$ sbt "compose architecture baseline platform zynq bitstream example.bd"
\end{lstlisting}
%
By default, the result of high-level synthesis will be placed in \code{cores}, the results of threadpool composition in \code{bd}.
\tpc{} will also look for the IP cores in the given directory and only build kernel-architecture combinations that are missing, example:
%
\begin{lstlisting}[language=bash]
~$ sbt "compose architecture baseline platform zynq bitstream example.bd coreDir myCores"
\end{lstlisting}
%
Finally, all previously mentioned command line arguments can also be moved to an external file (see \tblref{tbl:run-description}) called and referenced using the \code{configFile} argument.
E.g., the previously used commands can be moved into a file \code{myconfig.d} in \tpchome{} with the following contents:
%
\begin{lstlisting}[language=run]
Architecture = baseline
Platform = zynq
Bitstream = example.bd
\end{lstlisting}
%
And the execution can be started using
%
\begin{lstlisting}[language=bash]
~$ sbt "compose configFile myconfig.d"
\end{lstlisting}
%
A complete listing of all currently defined arguments can be found in \tblref{tbl:run-description}.
Unknown arguments will be ignored.
